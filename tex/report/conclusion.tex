\section{Conclusion}

Inverse planning is a viable option for computational research into insect navigation.
It can be used to distinguish between different models, to propose new experiments and to explore latent variables in a computational model.
Representation details, such as memory, can be hypothesized about and compared computationally, forcing model assumptions to be made explicit.
An advantage of using Bayesian methods in this frame is the possibility of conditionality on a variety of parameters, such as goal priors or model priors which could be taken directly from experiments, or hypothesized about to make predictions for future experiments.
