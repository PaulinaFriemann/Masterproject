The modeling of movement of biological agents is an active research field \cite<c.f. e.g.,>{lemoel2020a, webb2016}.
Computational models can support the discovery of biological mechanisms underlying a specific skill, help to discern between different theories, and propose new experiments to further develop existing theories.
Models can furthermore be used to make assumptions explicit, and discover limitations and restrictions to theories and methods.
Moreover, they can aid in exploring internal variables which cannot be measured directly, such as internal representations of perceived external information.

Many insights from biological agents' ways of solving problems have been used in research on Artificial Intelligence (AI; \citeNP{darwish2018}).
Theories, such as the path integration strategy found in some insects \cite<cf.,>{heinze2018b, wehner2009} have been applied for example in biologically-inspired mobile robots \cite<e.g.,>{lambrinos2000, weber1999}.
Oftentimes, these applications take inspiration from abstract perspectives on biologically implemented skills (for a recent overview, see \citeNP{darwish2018}, and \citeNP{agarwal2020}, for a review specifically for robot navigation).
These biologically-inspired algorithms are then optimized for a specific purpose, such as application, and computational or memory simplicity, and rather ignorant to the biological counterpart,
which is especially illustrated by the application of swarm optimization techniques for path-finding of single agents \cite<cf.,>{agarwal2020}, and hopes to hybridize different bio-inspired algorithms \cite{fan2020, darwish2018}.

In this project, a framework to explore the other direction of development will be implemented:
to build a bridge between AI and theory development in research on cognitive skills of biological agents, a framework is built with which computational methods from AI can be assessed on their adequacy to simulate or model biological agents.
The focus will be on navigation, however, the method can be extended to include any type of (conscious or unconscious) decision-based cognitive skill.
The framework can also be used to develop and test novel theories, to draw case-based probabilistic inferences to internal parameters, 
