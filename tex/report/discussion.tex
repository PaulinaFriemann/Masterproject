As mentioned in the results section, representation types (so a grid-based representation where the agent \textit{knows} its position in the environment versus a vector representation where the agent knows the vector towards its goal) cannot be distinguished in this frame,
as they are isomorphisms and essentially indistinguishable when combined with the way the algorithms work.
This is a very basic issue, and needs to be addressed in any future work on this framework.

Distinguishing the other factors however, algorithm type and $\beta$ values, is working as expected, but could be improved by developing methods to mathematically assimilate the interactions between $\beta$ and algorithm type.
This would make it easier to measure influences of these parameters independently.

Experiment 1 shows that distinguishing between algorithm type in case of the used algorithms is less reliable if no obstacles are present in the environment.
This is to be expected, as both algorithms would choose the direct path to the goal.

The experiments have shown that, as expected, obstacles constituted a problem for the greedy type of algorithm, which can be seen by the high number of necessary steps to reach the goal.
However, the high number of steps in Experiment 2a with the convex obstacle shows that due to the discreteness of the environment, even convex obstacles are a problem to this algorithm.

Recognition of $\beta$ values however works well, especially for Experiments 2a and 2b, as depicted in Figures \ref{fig:rbetaconvex} and \ref{fig:rbeta}.

Future work on this project holds a variety of possibilities.
The first step would be to improve the comparability of models.
There are facets to this: on the one hand, because of the desired and implemented modularity of the modeling approaches, interactions between the modules needs to be explored, 
to achieve meaningful interpretations of a module's details independent of the other modules in a model.

The other side to this is the impact of parameters, such as the determinism factor.
It needs to be explored how to avoid interactions between model specifications and parameter values, in order to be able to draw conclusions on the parameter itself and its effects.
A useful addition to the project for these means, and also for the implementation and development of other or novel approaches, would be a set of baseline experiments translated from real experiments.

From this baseline, other representation and navigation approaches can be explored.
More biologically plausible approaches should be implemented and analyzed, and their interactions studied. For this, existing models and insights from behavioral and neuro-science need to be consulted.

A further possibility would be to integrate a form of ``mental states'' into the models. This could be approached a variety of ways, such as integrating agent states into the environment, or allowing models to be dynamic over time.
The next step would then be to implement inferences to mental states.
E.g., if one wanted to infer whether an agent is in a state of hunger, or infer the amount of exploratory pressure.

Moving on to insect data however needs to include an exploration into as to how far the Markov Property holds for insect movement.
If this is not the case, other forms of problem description need to be taken into account.
A type of memory would in this case be needed, which expands past the heuristic itself.
